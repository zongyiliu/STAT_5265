\documentclass[letterpaper]{article} 
\usepackage[utf8]{inputenc}
\usepackage[T1]{fontenc}
\usepackage{amsmath}
\usepackage{amsfonts}
\usepackage{amssymb}
\usepackage{array}
\usepackage{hyperref}
\usepackage[version=4]{mhchem}
\usepackage{stmaryrd}
\usepackage[dvipsnames]{xcolor}
\colorlet{LightRubineRed}{RubineRed!70}
\colorlet{Mycolor1}{green!10!orange}
\definecolor{Mycolor2}{HTML}{00F9DE}
\usepackage{graphicx}
\usepackage{amsmath}
\usepackage{graphicx}
\usepackage{capt-of}
\usepackage{lipsum}
\usepackage{fancyvrb}
\usepackage{tabularx}
\usepackage{listings}
\usepackage[export]{adjustbox}
\graphicspath{ {./images/} }
\usepackage[utf8]{inputenc}
\usepackage[english]{babel}
\usepackage{float}
\usepackage{lipsum}
\usepackage{graphicx}
\usepackage{float}
\usepackage[margin=0.7in]{geometry}
\usepackage{amsmath}
\usepackage{graphicx}
\usepackage{capt-of}
\usepackage{tcolorbox}
\usepackage{lipsum}
\usepackage{graphicx}
\usepackage{float}
\usepackage{listings}
\usepackage{hyperref} 
\usepackage{xcolor} % For custom colors
\lstset{
	language=Python,                % Choose the language (e.g., Python, C, R)
	basicstyle=\ttfamily\small, % Font size and type
	keywordstyle=\color{blue},  % Keywords color
	commentstyle=\color{gray},  % Comments color
	stringstyle=\color{red},    % String color
	numbers=left,               % Line numbers
	numberstyle=\tiny\color{gray}, % Line number style
	stepnumber=1,               % Numbering step
	breaklines=true,            % Auto line break
	backgroundcolor=\color{black!5}, % Light gray background
	frame=single,               % Frame around the code
}
\usepackage{float}
\usepackage[]{amsthm} %lets us use \begin{proof}
	\usepackage[]{amssymb} %gives us the character \varnothing
	
	\title{Homework 3, MATH 5265}
	\author{Zongyi Liu}
	\date{Mon, Oct 24, 2025}
	\begin{document}
		\maketitle
		
		\section{Question 1 }
		
	Let us use the $O U$ process:
	
	$$
	d X_{t}=\kappa\left(\theta-r_{t}\right) d t+\sigma d W_{t}, \quad X_{0}=x_{0} \in \mathbb{R}
	$$
	
	where $W=\left(W_{t}\right)_{t \geq 0}$ is a $\mathbb{P}$-Brownian motion as a model for the USD/EUR exchange rate. For this purpose download 5 years worth of data from the St. Louis Fed: \href{https://fred.stlouisfed.org/series/DEXUSEU}{https://fred.stlouisfed.org/series/DEXUSEU}. The coefficients of the $O U$ process can be inferred from a Regression (AR) Model. To be explicit, by letting $\Delta=t_{i+1}-t_{i}$ a discretization of the $O U$ solution reveals that:
	
	$$
	X_{t_{i+1}}=X_{t_{i}} e^{-\kappa \Delta}+\theta\left(1-e^{-\kappa \Delta}\right)+\sigma \int_{t_{i}}^{t_{i+t}} e^{-\kappa\left(t_{i+t}-s\right)} d W_{s}
	$$
	
	Letting $\alpha=\theta\left(1-e^{-\kappa \Delta}\right), \beta=e^{-\kappa \Delta}$ and $\epsilon_{i}=\sigma \int_{t_{i}}^{t_{i+1}} e^{-\kappa\left(t_{i+1}-s\right)} d W_{s}$ so that $\gamma^{2}:=\operatorname{Var}\left(\epsilon_{i}\right)=\frac{\sigma^{2}}{2 \kappa}\left(1-e^{-2 \kappa \Delta}\right)$ we get:
	
	$$
	X_{t_{i+1}}=\alpha+\beta X_{t_{i}}+\epsilon_{i}
	$$
	
	where $\epsilon_{i} \stackrel{\text { i.i.d. }}{\sim} N\left(0, \gamma^{2}\right)$. Fit this regression model to the data using a programming language of your choice and report the implied values of $\kappa, \theta$ and $\sigma$ that you obtain by rearranging the equations for $\alpha, \beta$ and $\gamma$. Then, simulate a few trajectories from your calibrated model and compare them to the original price series.
	
		
		\textbf{Answer}
		
		\clearpage
		\section{Question 2}
		Consider the Vasicek model for the short rate $r= \left(r_{t}\right)_{t \geq 0}$ under $\mathbb{Q}$ :
		
		$$
		d r_{t}=\kappa\left(\theta-r_{t}\right) d t+\sigma d W_{t}, \quad r_{0}=c \in \mathbb{R}
		$$
		
		where $W=\left(W_{t}\right)_{t \geq 0}$ is a $\mathbb{Q}$-Brownian motion. Suppose that $c=0.055, \theta= 0.05, \kappa=2$ and $\sigma=0.03$. Using the formulas from class for the bond prices $P_{t}(T)$, plot the time $t=0$ yield curve corresponding to maturities $T \leq 10$.
		
		
		\textbf{Answer}
		

		\clearpage
		\section{Question 3}
		
		Consider the CIR model for the short rate $r= \left(r_{t}\right)_{t \geq 0}$ under $\mathbb{Q}$ :
		
		$$
		d r_{t}=\kappa\left(\theta-r_{t}\right) d t+\sigma \sqrt{r_{t}} d W_{t}, \quad r_{0}=c \in \mathbb{R}
		$$
		
		where $W=\left(W_{t}\right)_{t \geq 0}$ is a $\mathbb{Q}$-Brownian motion. Using the PDE approach from class, repeat the arguments from the Vasicek model to obtain the system of ODEs that characterizes the bond price under the CIR model. This coincides with the exercise from class.
		
		
		\textbf{Answer}
		
		\section{Question 4}
		Consider the Vasicek model for the short rate $r= \left(r_{t}\right)_{t \geq 0}$ under $\mathbb{Q}$ :
		
		$$
		\left\{d r_{t}=\kappa\left(\theta-r_{t}\right) d t+\sigma d W_{t}, \quad r_{0}=c \in \mathbb{R}\right.
		$$
		
		where $W=\left(W_{t}\right)_{t \geq 0}$ is a $\mathbb{Q}$-Brownian motion. Let $P_{t}(U)$ be the price of a $U$ maturity bond and consider the bond call option with payoff $\left(P_{T}(U)-K\right)_{+}$for some strike $K>0$ that is paid time $T<U$. In class we priced this option by using the $T$-maturity bond as a numeraire. By employing similar arguments, price the option using the $U$-maturity bond as a numeraire. Verify that you obtain the same solution.\\
		Hint: Find a way to introduce the auxilliary $\mathbb{Q}^{U}$-martingale $X_{t}=\frac{P_{t}(T)}{P_{t}(U)}$.
		
		\textbf{Answer}
		
		
		
	\end{document}
