\documentclass[10pt]{article}
\usepackage[utf8]{inputenc}
\usepackage[T1]{fontenc}
\usepackage{amsmath}
\usepackage{amsfonts}
\usepackage{amssymb}
\usepackage[version=4]{mhchem}
\usepackage{stmaryrd}
\usepackage{bbold}

\title{GU4265/GR5265 HOMEWORK \#4 }

\author{}
\date{}


\begin{document}
\maketitle
Fall 2025

Total Points: 50\\
Assignment Date: Wednesday, December 3rd.\\
Due Date: Friday, December 12th at 11:59pm (EST).

\section*{Instructions}
Read each question carefully and complete all the requirements. This grade for this assignment will be entirely completion based.\\
Submission Instructions: Submit a PDF file of your solutions to Canvas by the stated deadline. Scanned handwritten solutions in PDF format will be accepted but must be written neatly. It is your responsibility to ensure that the solutions are clear and legible. Late submissions will NOT be accepted!

\section*{Problem Set}
Problem 1 (25 Points). There is a stock in the market $S=\left(S_{t}\right)_{t \geq 0}$ with current price $S_{0}=100$ and a bank account $B=\left(B_{t}\right)_{t \geq 0}$ with interest rate $r=0.05$. Suppose that you are told that the current (i.e. time $t=0$ ) implied volatility surface in the market is given by the function:

$$
\sigma_{i m p}(K, T)=\left(\alpha_{0}+\alpha_{1} \sqrt{T}\right)\left[1+\left(\beta_{0}+\beta_{1} T\right)\left(K / S_{0}-1\right)^{2}\right]
$$

where $\alpha_{0}=0.2, \alpha_{1}=0.025, \beta_{0}=0.5, \beta_{1}=-0.1$ for all strikes $K \geq 0$ and maturity times $T=0 \ldots 5$.

\begin{enumerate}
  \item Plot the implied volatility surface for strikes $K=75 \ldots 125$ and maturity times $T=0 \ldots 5$.
  \item Using this implied volatility function and the Black-Scholes formula illustrate the time $t=0$ price of call options, $C(K, T)$, corresponding to the strikes $K=75 \ldots 125$ and maturity times $T=0 \ldots 5$.
  \item Using the Breeden-Litzenberger formula illustrate the implied probability distribution function for the stock price at time $T=2$.
  \item Using Dupire's Formula, calibrate a local volatility function $\sigma_{\text {loc }}(t, s)$ from the implied volatility surface and illustrate its values for $t=0 \ldots 5$ and $s=75 \ldots 125$.
  \item Using the calibration in (4) simulate 5 paths on $[0,2]$ from the local volatility model under $\mathbb{P}$,
\end{enumerate}

$$
d S_{t}=\mu S_{t} d t+\sigma_{l o c}\left(t, S_{t}\right) S_{t} d W_{t}, \quad S_{0}=100
$$

when $\mu=0.1$. You may use the time discretization $\Delta t=0.01$ with the approximation

$$
S_{t+\Delta t} \approx S_{t}+\mu S_{t} \Delta t+\sigma_{l o c}\left(t, S_{t}\right) S_{t} \sqrt{\Delta t} Z, \quad Z \sim N(0,1)
$$

Using your simulated paths, plot the trajectories of both $S$ and the volatility $\sigma_{\text {loc }}\left(t, S_{t}\right)$.

Note: You may use any programming language. You do not have to explicitly compute the derivatives in the Breeden-Litzenberger and Dupire formulas. Instead, you may use a finite difference approach:

$$
f^{\prime}(x) \approx \frac{f(x+h)-f(x)}{h}, \quad f^{\prime \prime}(x) \approx \frac{f(x+h)-2 f(x)+f(x-h)}{h^{2}} .
$$

It may be useful to define a helper function to do this in your code.\\
Problem 2 (25 Points). Consider the following market dynamics under $\mathbb{P}$ :

$$
\begin{cases}d S_{t}=\mu S_{t} d t+\sigma S_{t} d W_{t} & S_{0}=s \in(0, \infty) \\ d B_{t}=r B_{t} d t & B_{0}=1\end{cases}
$$

In this market we know the wealth process of a self-financing portfolio that invests $\nu=\left(\nu_{t}\right)_{t \geq 0}$ dollars in the stock is given by

$$
d X_{t}=\left[(\mu-r) \nu_{t}+r X_{t}\right] d t+\sigma \nu_{t} d W_{t}, \quad X_{0}=x
$$

Suppose that you want to solve the following power utility maximization problem for your wealth at some terminal time $T>0$ :

$$
\mathcal{U}(t, x)=\sup _{\nu} \mathbb{E}_{t, x}\left[\frac{X_{T}^{1-\gamma}}{1-\gamma}\right], \quad \gamma \in(0,1) .
$$

\begin{enumerate}
  \item Write down the HJB equation associated with this problem.
  \item Assuming $\partial_{x x} \mathcal{U}<0$ find the feedback form for $\nu$ and simplify the HJB equation.
  \item Using the ansatz $\mathcal{U}(t, x)=\varphi(t) \frac{x^{1-\gamma}}{1-\gamma}$ derive an ODE for $\varphi$ that can be used to determine $\mathcal{U}(t, x)$.
  \item Suppose that you now pay a running quadratic penalty for taking risk and investing in the stock. This means that your new objective is
\end{enumerate}

$$
\mathcal{V}(t, x)=\sup _{\nu} \mathbb{E}_{t, x}\left[\frac{X_{T}^{1-\gamma}}{1-\gamma}-\kappa \int_{t}^{T} \nu_{s}^{2} d s\right], \quad \gamma \in(0,1), \quad \kappa>0
$$

Repeat steps (1)-(2) for this new formulation. For this question you do not have to simplify to a system of ODEs.


\end{document}
