\documentclass[10pt]{article}
\usepackage[utf8]{inputenc}
\usepackage[T1]{fontenc}
\usepackage{amsmath}
\usepackage{amsfonts}
\usepackage{amssymb}
\usepackage[version=4]{mhchem}
\usepackage{stmaryrd}
\usepackage{bbold}
\usepackage{mathrsfs}

\title{GU4265/GR5265 Practice Problems }

\author{}
\date{}


%New command to display footnote whose markers will always be hidden
\let\svthefootnote\thefootnote
\newcommand\blfootnotetext[1]{%
  \let\thefootnote\relax\footnote{#1}%
  \addtocounter{footnote}{-1}%
  \let\thefootnote\svthefootnote%
}

%Overriding the \footnotetext command to hide the marker if its value is `0`
\let\svfootnotetext\footnotetext
\renewcommand\footnotetext[2][?]{%
  \if\relax#1\relax%
    \ifnum\value{footnote}=0\blfootnotetext{#2}\else\svfootnotetext{#2}\fi%
  \else%
    \if?#1\ifnum\value{footnote}=0\blfootnotetext{#2}\else\svfootnotetext{#2}\fi%
    \else\svfootnotetext[#1]{#2}\fi%
  \fi
}

\begin{document}
\maketitle
Fall 2025

Total Points: N/A\\
Assignment Date: Monday, December 8th.\\
Due Date: N/A

\section*{Problem Set}
Problem 1 (Ho-Lee Model). Suppose that the short rate, $r=\left(r_{t}\right)_{t \geq 0}$, in the market has the dynamics

$$
d r_{t}=\theta(t) d t+\sigma d W_{t}^{\mathbb{Q}}
$$

under the risk neutral measure $\mathbb{Q}$, where $\theta(t)$ is a deterministic function of time and $W^{\mathbb{Q}}$ is a $\mathbb{Q}$-Brownian motion. Derive the $T$-maturity zero coupon bond price in this model and state the SDE that the bond price satisfies under the measure $\mathbb{Q}$.

Hint: Recall the approach in the lectures for the Vasicek model.\\
Problem 2. Let $P_{t}(T)$ be the time $t$ price of a $T$-maturity zero coupon bond. The time $t$ forward rate spanning $T$ to $U(U>T)$ is given by

$$
\ell_{t}(U, T)=\frac{1}{U-T}\left[\frac{P_{t}(T)}{P_{t}(U)}-1\right]
$$

Suppose that there is a bank in the market that will allow you to secure (at no additional cost) a loan or investment today (i.e. $t=0$ ) that will take place in the future from time $T$ until time $U$ at the simple rate $F<\ell_{0}(U, T)$. That is, an investment/debt of $\$ 1$ at time $T$ will grow to $\$(1+F(U-T))$ at $U$. Construct an arbitrage portfolio by trading in the bonds and taking advantage of the offer provided by the bank.\\
Hint: How would you secure the forward rate for an investment at time $T$ by trading in the market?

Problem 3. Draw a stylized example of the "volatility smile" in equity markets for two different times to maturity $\tau_{1}<\tau_{2}$. Explain, in your own words, how the implied volatility surface is derived from market prices and discuss the phenomenon it illustrates.

Problem 4. Suppose that you would like to liquidate $q$ shares of a stock in the market with price, $S=\left(S_{t}\right)_{t \geq 0}$, and $\mathbb{P}$-dynamics,

$$
d S_{t}=-b \nu_{t} d t+\sigma d W_{t}, \quad S_{0}=s
$$

Here $\nu=\left(\nu_{t}\right)_{t \geq 0}$ is your trading rat $\varepsilon^{1}$ and $b>0$ reflects the permanent impact that you have on the price. Your inventory $Q=\left(Q_{t}\right)_{t \geq 0}$ satisfies

$$
d Q_{t}=-\nu_{t} d t, \quad Q_{0}=q
$$

We suppose that you also have a temporary impact on the price $\epsilon>0$ when you trade so that when you sell the (infinitesimal) quantity $\nu_{t} d t$ you only receive $\left(S_{t}-\epsilon \nu_{t}\right) \nu_{t} d t$ dollars. At some terminal time $T>0$ the marked-to-market value of your holdings is $Q_{T} S_{T}$. Since, your goal is to liquidate your initial position, we assume that you pay a quadratic terminal inventory penalty $\varphi\left(Q_{T}\right)^{2}>0$ on any left-over inventory at $T$. Therefore, by allowing for an arbitrary starting time $t<T$, price $S_{t}=s$, and starting inventory $q$, you are faced with the optimal control problem,

$$
\mathcal{U}(t, q, s)=\sup _{\nu \in \mathcal{A}} \mathbb{E}_{t, q, s}\left[\int_{t}^{T}\left(S_{u}-\epsilon \nu_{u}\right) \nu_{u} d u+Q_{T} S_{T}-\varphi\left(Q_{T}\right)^{2}\right]
$$

where $\mathcal{A}$ is the usual admissible set of controls from class.

\begin{enumerate}
  \item Using the fact that the generator $\mathcal{L}$ of the $2 D$-process $(Q, S)$ is,
\end{enumerate}

$$
\mathcal{L}=-\nu \partial_{q}-b \nu \partial_{s}+\frac{1}{2} \sigma^{2} \partial_{s s}
$$

write down the HJB equation for this control problem.\\
2. Using the structure of the HJB equation in $\nu$, solve for the optimal control in feedback form.\\
3. Substitute the feedback form back into the HJB equation and obtain a simplified (but still nonlinear) PDE for $\mathcal{U}(t, q, s)$.

\footnotetext{${ }^{1} \nu_{t}>0$ for selling, $\nu_{t}<0$ for buying.
}Hint: For (2) observe the functional form of the term under the $\sup \{\cdots\}$ and compare with our solution approach in class.

Problem 5. Suppose that there is a firm that extracts some natural resource (oil, gas, etc.) with price dynamics,

$$
d X_{t}=\mu X_{t} d t+\sigma X_{t} d W_{t}
$$

The running profit from extraction is given as a linear function of the current price, $\kappa X_{t}-\lambda$, for some $\kappa, \lambda>0$. At any time, the firm can choose to halt extraction and receive a fixed liquidation value (or cost) $K$. The firm would like to maximize the discounted (at some rate $\beta \geq 0$ ) expected value of their extraction operation. This can be cast as the following infinite horizon optimal stopping problem,

$$
V(t, x)=\sup _{\tau \in \mathcal{T}_{t, \infty}} \mathbb{E}_{t, x}\left[\int_{t}^{\tau} e^{-\beta(s-t)}\left(\kappa X_{s}-\lambda\right) d s+e^{-\beta(\tau-t)} K\right]
$$

\begin{enumerate}
  \item Argue that the value function does not need to depend on $t$.
  \item Write down the Dynamic Programming Equation (DPE).
  \item Write the "free boundary" ODE associated with the DPE in terms of the unknown stopping region $\mathscr{S}$ and continuation region $\mathscr{C}$.
\end{enumerate}


\end{document}
